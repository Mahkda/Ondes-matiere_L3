\documentclass[12pt,a4paper]{report}
\usepackage[utf8]{inputenc}
\usepackage[french]{babel}
\usepackage[T1]{fontenc}
\usepackage{amsmath}
\usepackage{amsfonts}
\usepackage{amssymb}
\usepackage{graphicx}
\author{Malo Kerebel}

\usepackage{wasysym}

\usepackage{systeme}

\usepackage{hyperref}
\hypersetup{
    colorlinks,
    citecolor=black,
    filecolor=black,
    linkcolor=black, urlcolor=black
}

\newcommand{\Ens}[1]{\mathbb{#1}}
\newcommand{\ens}[1]{\mathbb{#1}}
\newcommand{\fphi}{\quad \forall \varphi \in \mathbb{D}}

\input cyracc.def
\font\tencyr=wncysc10
\def\cyr{\tencyr\cyracc}
\def\dc{\mbox{\cyr SH}}


\begin{document}

\begin{titlepage}

\centering{
	
	{\scshape\LARGE Université de Bretagne Occidentale \par}
	\vspace{1cm}
	{\scshape\Large Note de cours\par}
	\vspace{1.5cm}
	{\huge\bfseries Ondes et Matière \unskip\strut\par}
	\vspace{2cm}
	{\Large\itshape Malo Kerebel \par}
	\vfill
	Cours par\par
	Bruno \textsc{Rouvellou}

	\vfill

% Bottom of the page
	{\large Semestre 6, année 2020-2021 \par}
}
\end{titlepage}

\tableofcontents

\chapter{Rappel d'ondes électromagnétiques dans le vide}

\begin{center}
\textbf{CM1 (2021-01-15)}
\end{center}

\section{Équations de Maxwell}

\begin{enumerate}
	\item \[
		\vec{\nabla} \cdot \vec{E} = div \vec{E} = \dfrac{\rho}{\varepsilon_0}
	\]
	\item \[
		\vec{\nabla} \cdot \vec{B} = div \vec{B} = 0
	\]
	\item \[
		\vec{\nabla} \wedge \vec{E} = rot \vec{E} = -\dfrac{\partial \vec{B}}{\partial t}
	\]
	\item \[
		\vec{\nabla} \wedge \vec{B} = rot \vec{B} = \mu_0 \left( \vec{j} + \varepsilon_0 \dfrac{\partial \vec{E}}{\partial t} \right)
	\]
\end{enumerate}

1) Forme locale du théorème de Gauss : \( \oiint_{(S)} \vec{E} \cdot d\vec{S} = \frac{q_{int}}{\varepsilon_0}\)

2) pas de monopole magnétique : \( \oiint_{(S)} \vec{B} \cdot d\vec{S} = 0\)

3) Induction de Faraday : \(\oint \vec{E} \cdot d\vec{l} = - \frac{d \Phi}{dt}\)

4) Théorème d'Ampère : \(\oint \vec{B} \cdot d\vec{l} = \mu_0I \Rightarrow rot \vec{B} = \mu_0 \vec{j}\)

\section{Ondes électromagnétiques dans le vide}

Les équations 1 et 4 deviennent :
\[
	div \vec{E} = 0 \quad \quad rot \vec{B} = \mu_0 \varepsilon_0 \dfrac{\partial \vec{E}}{\partial t}
\]
\(\rho\) et \(j\) sont nuls dans le vide ((absence de source de courant).

\subsection{Équations de propagation}

En prenant le rotationnel de l'équation (3) il vient :
\[
	rot (rot \vec{E}) = - rot \dfrac{\partial \vec{B}}{\partial t} = - \dfrac{\partial}{\partial t} rot \vec{B}
\]

À l'aide de l'équation 4' on obtient :
\[
	rot (rot \vec{E}) = - \mu_0 \varepsilon_0 \dfrac{\partial^2 \vec{E}}{\partial^2 t}
\]
\[
	rot (rot \vec{E}) = \vec{grad} (div \vec{E}) - \Delta \vec{E}
\]
compte tenu de 1' :
\[
	\Delta \vec{E} = \mu_0 \varepsilon_0 \dfrac{\partial^2 \vec{E}}{\partial^2 t}
\]

avec \(\mu_0\) la perméabilité du vide

De même pour B en utilisant 4', 3 et 2
On a l'équation de propagation :
\[
	\Delta \vec{B} = \mu_0 \varepsilon_0 \dfrac{\partial^2 \vec{B}}{\partial^2 t}
\]

\subsection{Onde plane progressive monochromatique}

\begin{align}
    \vec{E} &= \vec{E_m} \cos(\vec{k} \cdot \vec{r} - \omega t = \begin{bmatrix}
           E_x \\
           E_y \\
           E_z
         \end{bmatrix} \cos (\vec{k} \cdot \vec{r} - \omega t)\\         
    \vec{B} &= \vec{B_m} \cos(\vec{k} \cdot \vec{r} - \omega t = \begin{bmatrix}
           B_x \\
           B_y \\
           B_z
         \end{bmatrix} \cos (\vec{k} \cdot \vec{r} - \omega t)
  \end{align}

La divergence du champ électrique est donné par :
\[
	\vec{\nabla} \cdot \vec{E} = div \vec{E} = \dfrac{\partial E_x}{\partial x} + \dfrac{\partial E_y}{\partial y} + \dfrac{\partial E_z}{\partial z} = -(E_x k_x + E_y k_y + E_z k_z) \sin(\vec{k} \cdot \vec{r} - \omega t)
\]

La condition \(div \vec{E} = 0\) implique donc que \(\vec{E_m} \cdot \vec{k} = 0 \Rightarrow \vec{E}\) est transversal, de même \(div \vec{B} = 0 \Rightarrow \vec{B}\) est transversal.

À l'aide l'équation 3 on obtient :
\[
	\vec{k} \wedge \vec{E_m} = \omega \vec{B_m} \text{ ou } \vec{k} \wedge \vec{E} = \vec{B} 
\]

En notation complexe c'est plus simple.

\[
	div \vec{E} = i\vec{k} \cdot \vec{E} \quad \quad rot \vec{E} = i\vec{k} \wedge \vec{E}
\]
\[
	\dfrac{\partial \vec{E}}{\partial t} = -i \omega \vec{E} \quad \quad \Delta \vec{E} - k^2 \vec{E}
\]

\paragraph{relation de dispersion}

\[
	k = \dfrac{\omega}{c}
\]

\paragraph{Polarisation}

Hypotèse : propagation selon \(o_z \Rightarrow \vec{k}\cdot \vec{r} = kz\)
\begin{align*}
	E_x &= E_{m_0x} \cos (kz - \omega t)\\
	E_y &= E_{m_0y} \cos (kz - \omega t + \phi)\\
	E_z &= 0
\end{align*}

Par définition ce qui définit la polarisation est la direction de E.

Les composantes du champ électriques peuvent s'écrire sous la forme :
\[
	\dfrac{E_x}{E_{m_0x}} = \cos (kz - \omega t)
\]
\[
	\dfrac{E_y}{E_{m_0y}} = \cos (kz - \omega t) \cos \phi -\sin (kz - \omega t) \sin \phi
\]

d'où après simplification : 
\[
	\dfrac{E_y^2}{E^2_{m_0y}} + \dfrac{E_x^2}{E^2_{m_0x}} = \dfrac{2 E_y}{E^2_{m_0x}} \dfrac{E_y}{E_{m_0y}}\cos \phi + \sin^2 \phi
\]

Dans le cas \(\phi = 0 ou \phi = \pi\) on a une polarisation rectiligne.
Si on a \(\phi \pm \pi/2\) on a une polarisation circulaire.

\subsection{Vecteur de Poynting}

Charge élémentaire \(\rho d \tau\) (Charge \(\rho\) dans un volume élémentaire \(d\tau\)) subie dans oem : \(\vec{F} = \rho d \tau (\vec{E} + \vec{v} \wedge \vec{B})\)

La puissance fourni à cette charge élémentaire est :
\begin{align*}
	\vec{F} \cdot \vec{v} &= \rho d \tau (\vec{E} + \vec{v} \wedge \vec{B}) \cdot \vec{v}\\
	&=  \rho d \tau \vec{E} \cdot \vec{v}
\end{align*}
On peut réécrire :
\[
	\vec{F} \cdot \vec{v} = \vec{j} \cdot \vec{E}d\tau
\]
(j la densité du courant)
avec la 4ème équation de Maxwell on a :
\[
	\vec{E}\cdot \vec{j} = \dfrac{1}{\mu_0} rot \vec{B} \cdot \vec{E} - \epsilon_0 \vec{E} \cdot \dfrac{\partial \vec{E}}{\partial t}
\]
En utilisant :
\[
	div(\vec{U} \wedge \vec{V}) = rot (\vec{U}) \cdot \vec{V} - \vec{U} \cdot rot \vec{V}
\]
Ainsi, on a :
\[
	\vec{E} \cdot \vec{j} = -div \left( \vec{E} \wedge \dfrac{\vec{B}}{\mu_0} \right) + \dfrac{\vec{B}}{\mu_0} \cdot rot \vec{E} - \varepsilon_0 \vec{E} \cdot \dfrac{\partial \vec{E}}{\partial t}
\]

À l'aide de la troisième équation de Maxwell 
\[
	\vec{E} \cdot \vec{j} = -div \left( \vec{E} \wedge \dfrac{\vec{B}}{\mu_0} \right) - \dfrac{\vec{B}}{\mu_0} \cdot rot \vec{E} - \varepsilon_0 \vec{E} \cdot \dfrac{\partial \vec{E}}{\partial t} 
\]
On définit le vecteur de Poynting R par la relation :
\[
	\vec{R} = \vec{E} \wedge \dfrac{\vec{B}}{\mu_0}
\]

\[
	\vec{E} \cdot \vec{j} = -div \vec{R} - \dfrac{\partial}{\partial t} w
\]
avec \(w = \frac{1}{2} \varepsilon_0 E^2 + \frac{1}{2\mu_0} B^2\)

En abscence de courant on a la loi de conservation des charges :
\[
	div \vec{R} + \dfrac{\partial}{\partial t} w = 0
\]
On peut considérer le vecteur de Poynting cocmme une "densité de courant d'énergie". Il est de la forme :
\[
	\vec{R} = w \vec{v}
\]
où \(\vec{v}\) est la vitesse de propagation de l'énergie

Le vecteur de Poynting a donc pour expression :
\[
	\vec{R} = \vec{E} \wedge \dfrac{(\vec{k} \wedge \vec{E})}{\omega \mu_0}
\]

Finalement :
\[
	\vec{R} = \dfrac{k E^2}{\omega \mu_0} \dfrac{\vec{k}}{k}
\]

La densité d'énergie magnétique peut s'écrire sous la forme :
\[
	W_m = \dfrac{1}{2} \dfrac{B^2}{\mu_0} = \dfrac{1}{2} \varepsilon_0 c^2 B^2
\]

On peut écrire :
\[
	E = \dfrac{\omega}{k} B = cB
\]

La densité d'énergie magnétique est donc égale à la densité d'énergie électrique

\subsection{Impédance cacractéristique}

Dans le cas d'une onde plane monochromatique on définit l'impédance caractéristique d'un milieu par :
\[
	Z = \dfrac{E}{H}
\]

Dans le vide on définit H par :
\[
	\vec{B} = \mu_0 \vec{H}
\]

Donc l'impédance du vide est de 
\[
	Z_0 = 377 \Omega
\]

\chapter{Ondes et matières dans des milieu linéaire homogène isotrope}

\begin{center}
\textbf{CM2 (2021-01-20)}
\end{center}

\section{Milieux dispersifs}

C'est un milieu matériel, transparent aux ondes électro-magnétiques
Les ondes planes progressives (O.P.P) restent solutions de l'équation de propagation \(\neq\) de celle du vide.

Les conditions à respecter sont différentes, nouvelles relation de dispersion

\paragraph{Ex:} O.P.P.M

Propagation suivant \(O_z\) et polarisation suivant \(O_y\)
\[
	E_y = E_m e^{i(kz-\omega t)}
\]

\paragraph{Def :}
La relation de dispersion est la relation h(\(\omega, \omega(h)\) entre la pulsation et le nombre d'onde

\paragraph{Def}
La vitesse de phase est la vitesse de propagation des plans équiphases

\begin{align*}
	\Rightarrow E_y = E_m e^{i(kz - \omega t)} &=  E_m e^{ik(z - \frac{\omega}{k} t)}\\
	&= E_m e^{ik(z-v_\phi t)}
\end{align*}

\[
	v_\phi = \dfrac{\omega}{k}
\]

- Dans le vide \(v_\phi\) est indépendant de la pulsation \(\omega\) \(v_\phi = c\)

Dans le vide, \(k = \frac{\omega}{c}\)

- Dans la matière \(v_\phi\) dépend en général de \(\omega\)

\paragraph{Def :} un milieu dispersif est un milieu dans lequel la vitesse de phase dépend de la pulsation

\paragraph{Def :} indice, \( v_\phi = \dfrac{c}{n(\omega)}\)

\(\Rightarrow\) Conséquence : Superposition d'OPPM de différente pulsation, n'est pas en générale une OPPM car les ondes ne se propagent pas à la même vitesse.

\subsection{E : Somme de 2 OPPM}

\[
	\Rightarrow E_y = E_m \left( e^i(k_1 x - \omega_1 t) + e^i(k_2 x - \omega_2 t) \right)
\]

avec :
\begin{align*}
	k_1 &= k_0 - \frac{\Delta k}{2}\\
	k_2 &= k_0 + \frac{\Delta k}{2}\\
	\omega &= \omega(k)\\
	\Delta k &= k_2 - k_1 \ll k_0
\end{align*}

On fait un DL1 autour de \(k_0\) : \(\left( \omega'(k_0) = \frac{d\omega}{d k}_{k_0} \right)\)

\begin{align*}
	\omega_1 &= \omega (k_1) \simeq \omega(k_0) - \frac{\Delta k}{2} \omega' (k_0) = \omega_0 - \frac{\Delta k}{2} \omega' (k_0)\\
	\omega_2 &= \omega (k_2) \simeq \omega(k_0) - \frac{\Delta k}{2} \omega' (k_0) = \omega_0 - \frac{\Delta k}{2} \omega' (k_0)
\end{align*}

\[
	\Rightarrow E_y = E_m e^{i(k_0 x - \omega t)}  \left[ e^{i(- \frac{\Delta k}{2} x + (\frac{\Delta k}{2} \omega'(k_0))t)} + e^{i(+ \frac{\Delta k}{2} x - (\frac{\Delta k}{2} \omega'(k_0))t)} \right]
\]

\[
	\Rightarrow E_y = 2 E_m e^{i(k_0 x - \omega_0 t} \cos(\frac{\Delta k}{2} (x - \omega'(k_0)t))
\]
\[
	\Rightarrow E_y = Re(E_z)
\]
\[
	\Rightarrow E_y = 2 E_m \cos (k_0(x- \frac{\omega_0}{k_0} t)) \cos(\frac{\Delta k}{2}(x - \frac{\omega_0}{k_0} t))
\]

Produit de 2 OPPM qui se propagent à deux vitesses différentes
\[
	v_1 = \frac{\omega_0}{k_0} \quad \quad \quad v_2  \omega' (k_0) = \frac{d\omega}{dk}_{k = k_0}
\]

On obtient une onde modulé, avec une onde porteuse et une moduleuse.

\[
	v_1 = \frac{\omega_0}{k_0} = v_\phi \quad \quad v_2 = \frac{d\omega}{dk}_{k_0} = v_g \text{ vitesse de groupe}
\]

\(v_g\) vitesse de déplacement de l'enveloppe (modulation ici)

En terme de signaux \(\equiv \) vitesse de transmission de l'info
En terme d'énergie \(\equiv \) vitesse de transport de l'énergie 

Dans le vide 
\[
	v_g = \frac{d\omega}{dk} = \frac{dkc}{dk} = c = v_\phi
\]

Dans un milieu dispersif
\[
	v_g \neq v_\phi
\]

\subsection{Généralisation : paquets d'ondes}

En mécanique quantique ça permet de rendre la dualité onde-corpuscule possible

\[
	E_y = \int_{-\infty}^{+\infty} A(k) e^{i(kz-\omega t)} dk
\]
Somme d'une infinité d'onde, plutot que juste 2 dans l'exemple précedant. On prendra généralement \(A(k)\) une gaussienne.

\[
	\text{On prendra } \omega = \omega(k_0) + (k - k_0) \frac{d\omega}{dk}_{k_0}
\]

On transforme \(E_y\) en :
\begin{align*}
	E_y &= \int_{-\infty}^{+\infty} A(k) \exp(-\omega(k_0)t + k_0 z + (k - k_0)z - t(k - k_0)\frac{d\omega}{dk}_{k_0}) dk\\
	E_y &= e^{i(k_0 z - \omega(k_0)t} \int_{-\infty}^{+\infty} A(k) \exp \left(i \left[ (k - k_0)(z - \frac{d\omega}{dk}_{k_0}t) \right]\right) dk
\end{align*}

Dans un milieu non dispersif, les vitesses sont les même et donc le paquet d'onde n'est pas déformé, dans un milieu dispersif, les vitesses sont différentes et donc le paquet d'onde se déforme.

\paragraph{Remarque :} Souvent A(k) est une gaussienne on parle de paquet d'onde gaussien

\section{Polarisation de la matière}

On définit le moment dipolaire (ou polarisation) \(\vec{P}\) :
\[
	\vec{P} = N \vec{p}
\]
avec N, le nombre de molécule par unité de volume et \(\vec{p}\) le moment dipolaire moyen par molécule.

a) Potentiel créé par les dipoles d'un dipole électrique.
\[
	V_2 = \dfrac{\vec{p}\cdot \vec{r_1}}{4\pi \varepsilon_0 r^2}
\]

Dans un volume \(\tau'\), le volume élementaire \(d\tau'\), autour du point A, M le point où l'on calcule le potentiel.
\[
	dV(m) = \dfrac{1}{4\pi \varepsilon_0} \dfrac{\vec{P} \cdot \vec{r_1}}{r_M^2} d\tau'
\]

\begin{align*}
	dV(m) &= \dfrac{-1}{4\pi \varepsilon_0} \left( \vec{p} \cdot \vec{\nabla} (\frac{1}{r_m}) \right) d\tau'\\
	dV(m) &= \dfrac{1}{4\pi \varepsilon_0} \left( \vec{p} \cdot \vec{\nabla'} (\frac{1}{r_m}) \right) d\tau'
\end{align*}

Avec :
\[
	\vec{\nabla'} (\frac{1}{r}) = -\frac{\vec{-r_1}}{r^2} = \frac{\vec{r_1}}{r^2}
\]
Calculé en A pour permettre l'intégration sur le volume \(\tau'\), on a :
\[
	\vec{\nabla'} (\frac{1}{r}) = \frac{\vec{r_1}}{r^2}
\]
On a \(r \gg \) distance de la molécule \( \Rightarrow \) approximation dipolaire par intégration
On intègre donc \(dV(m)\) :
\[
	V(m) = \dfrac{1}{4\pi \varepsilon_0} \int_{\tau'} \vec{p} \cdot \vec{\nabla'} (\frac{1}{r_m}) d\tau'
\]
Or \(\vec{\nabla} (f \vec{A}) = f \vec{\nabla} \vec{A} + \vec{A} \cdot \vec{\nabla} f \)\\
Donc avec \(\vec{A} = \vec{p} \quad f = \frac{1}{r}\):
\[
	V(m) = \dfrac{1}{4\pi \varepsilon_0} \left( \int_{\tau'} \left( \vec{\nabla} \cdot \dfrac{\vec{p}}{r} \right) d\tau' - \int_{\tau'} \dfrac{\vec{\nabla} \cdot \vec{p}}{r} d\tau'\right)
\]
Par Green-ostrogradski :
\[
	V(M) = \dfrac{1}{4\pi \varepsilon_0} \left( \oint_{S'} \dfrac{\vec{p} \cdot \vec{dS'}}{r} - \int_{\tau'} \dfrac{\vec{\nabla} \cdot \vec{p}}{r} d\tau' \right)
\]
Avec S', la surface qui entoure \(\tau'\) dirigé vers l'extérieur

On définit :
\begin{align*}
	\sigma_{pol} &= \vec{p} \cdot \vec{n}\\ 
	\rho_{pol} &= - \vec{\nabla'} \cdot \vec{p}
\end{align*}

\[
	V(M) = \dfrac{1}{4\pi \varepsilon_0} \left( \oint_{S'} \dfrac{\sigma_{pol} \cdot \vec{dS'}}{r} - \int_{\tau'} \dfrac{\rho_{pol}}{r} d\tau' \right)
\]

\(\sigma_{pol}\) est la densité surfacique de chage liées et \(\rho_{pol}\) est la densité volumique de charge liées.

\begin{center}
\textbf{CM3 (2021-01-25)}
\end{center}

\(d\tau' = \vec{d} \cdot \vec{dS}\), Les charges qui vont passer à travers dS, s'obtient avec :\[
	dQ = N Q_m \vec{d} \cdot \vec{dS'}
\]
Avec \(N = \) Nombre de molécule par 
Charge traversant dS' sous l'action de E
\[
	dQ = N \vec{p} \cdot \vec{dS'} = \vec{P} \cdot \vec{dS'} = (\vec{P} \cdot \vec{n}) \vec{dS'}
\]
\[
	\sigma_{pol} = \dfrac{dQ}{dS'} = \vec{P} \cdot \vec{n}
\]

La charge totale qui a traversé S' est :
\[
	Q = \int_{S'} dQ = \int_{S'} \vec{P} \cdot \vec{dS'}
\]

Charge restante à l'intérieur de \(\tau'\) : \(-Q = \int_{\tau'} \rho_{pol} d\tau'\)
Par Green OStrogradski on a :
\[
	-Q = - \int \vec{\nabla} \cdot \vec{P} d\tau'
\]
Donc :
\[
	P_{pol} = - \vec{\nabla} \cdot \vec{P}
\]

c) Généralisation

* Le calcul de V à l'extérieur du milieu est calculé identiquement à l'intérieur

* E est créé par la distribution extérieure de charge à l'origine de (pas réussi à lire)

\[
	V_{tot} = \dfrac{1}{4\pi \varepsilon_0} \iiint_{\tau'} \dfrac{\rho_{libre} + \rho_{pol}}{r} d\tau' + \dfrac{1}{4\pi \varepsilon_0} \iint_{S'} \dfrac{\sigma_{libre} + \sigma_{pol}}{r} dS'
\]
\[
	\Rightarrow \vec{E} = \dfrac{1}{4\pi \varepsilon_0} \iiint_{\tau'} \dfrac{\rho_{libre} + \rho_{pol}}{r^2}\vec{r_1} d\tau' + \dfrac{1}{4\pi \varepsilon_0} \iint_{S'} \dfrac{\sigma_{libre} + \sigma_{pol}}{r^2} \vec{r_1} dS'
\]

(On écrira ensuite \(\sigma_l / \rho_l\) avec le l pour libre et \(\sigma_{pol} / \rho_{pol}\) pour polarisation (ou les charges liées)

d) Densité de courant de polarisation

Si \(E = E(t) \Rightarrow\) mouvement des charges liées \(\Rightarrow \vec{j_{pol}}\)

Conservation des charges donc :
\[
	\iint_S \vec{j_{pol}} \cdot \vec{dS} = - \dfrac{\partial}{\partial t} \iiint_\tau \rho_{pol} d\tau
\]
Car la vitesse des charges qui traverse S est égal à la vitesse de diminution à l'inter**** de S (\(\tau\)), par green Ostrogradski on a :
\[
	\iiint_\tau \vec{\nabla} \cdot \vec{j_{pol}} = - \dfrac{\partial}{\partial t} \iiint_\tau - (\vec{\nabla} \cdot \vec{P}) d\tau
\]
\[
	= \iiint_{\tau} \vec{\nabla} \cdot \dfrac{\partial \vec{P}}{\partial t} d\tau
\]
\[
	\Rightarrow \vec{j_{pol}} = \dfrac{\partial \vec{P}}{\partial t} (\text{Ampère}/m^2)
\]

\subsection{Polarisation magnétique et Aimantation}

\paragraph{Définitions}\quad \par

Diamagnétique, phénomène très faible, induit, moment magnétique compensé

Paramagnétique, phénomène faible, orientation des molécules, moment magnétique non compensé, l'intensité du phénomène est inversement proportionelle à la température (les molécules sont plus désorganisé à haute température)

Ferromagnétique, phénomène fort, phénomène collectif, persistant, non linéaire

\paragraph{Identification} entre magnétique et électrique : \par 

Une boucle de courant génère un champ magnétique, dont l'intensité varie suivant : \(\vec{m} = IS \vec{k}\)

En électrique on a la polarisation : \(\vec{P} = N \vec{p}\) en magnétique on a similairement : \(\vec{M} = N \vec{m}\) (qu'on appele plutot aimantation).

Le moment dipolaire : \(V_{dip} = \dfrac{\vec{p} \cdot \vec{r_1}}{4 \pi \varepsilon_0 r^2} \) en magnétique on a : \(\vec{A_{dip}} = \dfrac{\mu_0}{4\pi} \dfrac{\vec{m} \wedge \vec{r_1}}{\sigma^2}\) (On remarquera que c'est désormais un vecteur et non un scalaire)

\(dV_{dip} = \dfrac{\vec{R} \cdot \vec{r_1}}{4 \pi \varepsilon_0 r^2} \) et en magnétique : \(\vec{dA_{dip}} = \dfrac{\mu_0}{4\pi} \dfrac{\vec{M} \wedge \vec{r_1}}{\sigma^2}\)

Densité de charge surfacique en électrique : \(\sigma_{pol} = \vec{P} \cdot \vec{n}\) en magnétique la densité linéique est : \(\vec{\lambda_m} = \vec{M} \wedge \vec{n}\)

Et de même en dimension supérieur : \(\rho_{pol} = - \vec{\nabla} \cdot \vec{{P}}\) en magnétique : \(\vec{j_m} = rot \vec{M}\)

\section{Équation de Maxwell dans les milieux linéaires homogènes isotropes}

Deux équation sont des équation de structure et ne seront donc pas modifié

\[
	div \vec{B} = 0 \quad \quad rot \vec{E} = \dfrac{-\partial \vec{B}}{\partial t}
\]

La première équation va changer :
\[
	div \vec{E} = \dfrac{\rho}{\varepsilon_0} \quad \rho = \rho_l + \rho_{pol}
\]

Avec \(\rho_{pol} = - div \vec{P}\)

\begin{align*}
	&\Leftrightarrow \varepsilon_. div \vec{E} = \rho_l - div \vec{P}\\
	&\Leftrightarrow div (\varepsilon_0 \vec{E} + \vec{P}) = \rho_l
\end{align*}

On note donc \(\vec{D} = \varepsilon_0 \vec{E} + \vec{P}\) l'induction électrique 
\[
	\Leftrightarrow div \vec{D} = \rho_l
\]

Si le milieu est un LMHI on a 
\[
	\vec{p} = \propto \vec{E_{loc}}
\]

\[
	\Rightarrow \vec{P} = N \vec{p} = \varepsilon_0 \chi_e \vec{E}
\]

Avec \(\chi_e\) la suceptibilité électrique, la capacité du milieu a réagir à un milieu électrique.

\[
	\Rightarrow \vec{D} = \varepsilon_0 \vec{E} + \vec{P} = \varepsilon_0 (1 + \chi_e) \vec{E}
\]
Donc dans un MLHI on a \(\vec{D} = \varepsilon \vec{E}\) avec \(\varepsilon = \varepsilon_0 (1+\chi_e) = \varepsilon_0 \varepsilon_r\)

\paragraph{Remarque} Dans un MLHI \(\vec{D} \propto \vec{E}\)

La quatrième équation de Maxwell va changer aussi.
\[
	rot \vec{B} = \mu_0 \left( \vec{j} + \varepsilon_0 \dfrac{\partial \vec{E}}{\partial t}\right)
\]
Dans un MLHI on aura :
\[
	\vec{j} = \vec{j_l} + \vec{j_m} + \vec{j_{pol}}
\]
La densité de courant lié aux charges libres et aux charges liées
Avec \(\vec{j_{pol}} = \dfrac{\partial \vec{P}}{\partial t}\) et \(\vec{j_m} = rot \vec{M}\), donc :
\[
	rot \vec{B} = \mu_0 \left( \vec{j_l} + rot \vec{M} + \dfrac{\partial \vec{P}}{\partial t} + \varepsilon_0 \dfrac{\partial \vec{E}}{\partial t}\right)
\]
On définit le champ magnétique \(\vec{H} = \dfrac{\vec{B}}{\mu_0} - \vec{m}\)

Comme avant, dans un MLHI on a \(\vec{M} = \chi_m \vec{H} \Rightarrow \vec{H} = \dfrac{\vec{B}}{\mu_0} - \chi_m \vec{H}\), avec encore une fois, \(\chi_m\) un scalaire, la suceptibilité magnétique.
\[
	\vec{B} = \mu_0 ( 1 + \chi_m) \vec{H} = \mu \vec{H}
\]
Avec \(\mu = \mu_0 ( 1 + \chi_m) = \mu_0 \mu_r\)

\paragraph{Remarque} Si le matériau est transparent on prendra \(\chi_r = 1\) et donc \(\vec{B} \simeq \mu_0 \vec{H}\)

Dans un matériau diamagnétique on a \(\chi_m\) est inférieur à 0 et est indépendant de la température, pour le paramagnétique \(\chi_m\) est inversement proportionnel à la température et est supérieur à 0 et pour un matériau ferromagnétique, \(\chi_m\) augmente quand la température diminue et est \(\gg 0\), il y a existence de la température de Curie, en dessous de laquelle un matériau est ferromagnétique et la valeur de \(\chi_m\) dépend de la valeur de H, c'est donc non linéaire.

Dans un MLHI on peut obtenir : 
\[
	rot \vec{H} = \vec{j_l} + \dfrac{\partial \varepsilon \vec{E}}{\partial t} = \vec{j_l} + \dfrac{\partial \vec{D}}{\partial t}
\]

\begin{align*}
	M1 &: div \vec{E} = \dfrac{\rho_l}{\varepsilon} \Leftrightarrow div \vec{D = \rho_l}\\
	M2 &: div \vec{B} = 0\\
	M3 &:  rot \vec{E} = - \dfrac{\partial \vec{B}}{\partial t}\\
	M4 &: rot \vec{H} = \vec{j_l} + \dfrac{\partial \vec{D}}{\partial t}
\end{align*}

\section{Modèle de Drude-Lorrentz}

Hypothèse de départ, l'électron est lié aux atomes par une force de type ressort (oscillateur), la réponse de l'oscillateur détermine la polarisation de l'atome.\\
Modèle de DL :

\begin{itemize}
	\item L'électron a une charge -e et une masse m, une vitesse \(\vec{v}\) et une position \(\vec{r}\)
	\item Force de Lorrentz : \(-e (\vec{E} + \vec{v} \wedge \vec{B})\)
	\item Force de frottement visqueux(collision avec son environnement) : \(- \alpha \vec{v}\)
	\item Force de rappel (interaction des noyaux) : \(- k \vec{r}\)
\end{itemize}

Dans la force de Lorrentz la force magnétique est négligé sauf si on rajoute une force magnétique extérieur

\end{document}