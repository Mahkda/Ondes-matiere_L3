\documentclass[12pt,a4paper]{report}
\usepackage[utf8]{inputenc}
\usepackage[french]{babel}
\usepackage[T1]{fontenc}
\usepackage{amsmath}
\usepackage{amsfonts}
\usepackage{amssymb}
\usepackage{graphicx}
\author{Malo Kerebel}

\usepackage{wasysym}

\usepackage{systeme}

\usepackage{hyperref}
\hypersetup{
    colorlinks,
    citecolor=black,
    filecolor=black,
    linkcolor=black, urlcolor=black
}

\newcommand{\Ens}[1]{\mathbb{#1}}
\newcommand{\ens}[1]{\mathbb{#1}}
\newcommand{\fphi}{\quad \forall \varphi \in \mathbb{D}}

\input cyracc.def
\font\tencyr=wncysc10
\def\cyr{\tencyr\cyracc}
\def\dc{\mbox{\cyr SH}}


\begin{document}

\begin{titlepage}

\centering{
	
	{\scshape\LARGE Université de Bretagne Occidentale \par}
	\vspace{1cm}
	{\scshape\Large Note de cours\par}
	\vspace{1.5cm}
	{\huge\bfseries Ondes et Matière \unskip\strut\par}
	\vspace{2cm}
	{\Large\itshape Malo Kerebel \par}
	\vfill
	Cours par\par
	Bruno \textsc{Rouvellou}

	\vfill

% Bottom of the page
	{\large Semestre 6, année 2020-2021 \par}
}
\end{titlepage}

\tableofcontents

\chapter{Rappel d'ondes électromagnétiques dans le vide}

\begin{center}
\textbf{CC1 (2021-01-15)}
\end{center}

\section{Équations de Maxwell}

\begin{enumerate}
	\item \[
		\nabla \cdot E = div E = \dfrac{\rho}{\varepsilon_0}
	\]
	\item \[
		\nabla \cdot B = div B = 0
	\]
	\item \[
		\nabla \wedge E = rot R = -\dfrac{\partial B}{\partial t}
	\]
	\item \[
		\nabla \wedge B = rot B = \mu_0 \left( j + \varepsilon_0 \dfrac{\partial E}{\partial t} \right)
	\]
\end{enumerate}

1) Forme locale du théorème de Gauss : \( \oiint_{(S)} E \cdot dS = \frac{q_{int}}{\varepsilon_0}\)

2) pas de monopole magnétique : \( \oiint_{(S)} B \cdot dS = 0\)

3) Induction de Faraday : \(\oint E \cdot dl = - \frac{d \Phi}{dt}\)

4) Théorème d'Ampère : \(\oint B \cdot dl = \mu_0I \Rightarrow rot B = \mu_0 j\)

\section{Ondes électromagnétiques dans le vide}

Les équations 1 et 4 deviennent :
\[
	div E = 0 \quad \quad rot B = \mu_0 \varepsilon_0 \dfrac{\partial E}{\partial t}
\]
\(\rho\) et \(j\) sont nuls dans le vide ((absence de source de courant).

\subsection{Équations de propagation}

En prenant le rotationnel de l'équation (3) il vient :
\[
	rot (rot E) = - rot \dfrac{\partial B}{\partial t} = - \dfrac{\partial}{\partial t} rot B
\]

À l'aide de l'équation 4' on obtient :
\[
	rot (rot E) = - \mu_0 \varepsilon_0 \dfrac{\partial^2 E}{\partial^2 t}
\]
\[
	rot (rot E) = grad (div E) - \Delta E
\]
compte tenu de 1' :
\[
	\Delta E = \mu_0 \varepsilon_0 \dfrac{\partial^2 E}{\partial^2 t}
\]

avec \(\mu_0\) la perméabilité du vide

De même pour B en utilisant 4', 3 et 2
On a l'équation de propagation :
\[
	\Delta B = \mu_0 \varepsilon_0 \dfrac{\partial^2 B}{\partial^2 t}
\]

\subsection{Onde plane progressive monochromatique}

\begin{align}
    E &= E_m \cos(kr - \omega t = \begin{bmatrix}
           E_x \\
           E_y \\
           E_z
         \end{bmatrix} \cos (kr- \omega t)\\         
    B &= B_m \cos(kr - \omega t = \begin{bmatrix}
           B_x \\
           B_y \\
           B_z
         \end{bmatrix} \cos (kr- \omega t)
  \end{align}

La divergence du champ électrique est donné par :
\[
	\nabla E = div E = \dfrac{\partial E_x}{\partial x} + \dfrac{\partial E_y}{\partial y} + \dfrac{\partial E_z}{\partial z} = -(E_x k_x + E_y k_y + E_z k_z) \sin(kr - \omega t)
\]

La condition \(div E = 0\) implique donc que \(E_m k = 0 \Rightarrow \) E est transversal, de même B est transversal.

À l'aide l'équation 3 on obtient :
\[
	k \wedge E_m = \omega B_m \text{ ou } k \wedge E = B 
\]

En notation complexe c'est plus simple.

\[
	div E = ik \cdot E \quad \quad rot E = ik \wedge E
\]
\[
	\dfrac{\partial E}{\partial t} = -i \omega E \quad \quad \Delta E - k^2 E
\]

\paragraph{relation de dispersion}

\[
	k = \dfrac{\omega}{c}
\]

\paragraph{Polarisation}

Hypotèse : propagation selon \(o_z \Rightarrow k\cdot r = kz\)
\begin{align*}
	E_x &= E_{m_0x} \cos (kz - \omega t)\\
	E_y &= E_{m_0y} \cos (kz - \omega t + \phi)\\
	E_z &= 0
\end{align*}

Par définition ce qui définit la polarisation est la direction de E.

Les composantes du champ électriques peuvent s'écrire sous la forme :
\[
	\dfrac{E_x}{E_{m_0x}} = \cos (kz - \omega t)
\]
\[
	\dfrac{E_y}{E_{m_0y}} = \cos (kz - \omega t) \cos \phi -\sin (kz - \omega t) \sin \phi
\]

d'où après simplification : 
\[
	\dfrac{E_y^2}{E^2_{m_0y}} + \dfrac{E_x^2}{E^2_{m_0x}} = \dfrac{2 E_y}{E^2_{m_0x}} \dfrac{E_y}{E_{m_0y}}\cos \phi + \sin^2 \phi
\]

Dans le cas \(\phi = 0 ou \phi = \pi\) on a une polarisation rectiligne.
Si on a \(\phi \pm \pi/2\) on a une polarisation circulaire.

\subsection{Vecteur de Poynting}

Charge élémentaire \(\rho d \tau\) (Charge \(\rho\) dans un volume élémentaire \(d\tau\)) subie dans oem : \(F = \rho d \tau (R + v \wedge B)\)

La puissance fourni à cette charge élémentaire est :
\begin{align*}
	F \cdot v &= \rho d \tau (R + v \wedge B) \cdot v\\
	&=  \rho d \tau E \cdot v
\end{align*}
On peut réécrire :
\[
	F \cdot v = j \cdot Ed\tau
\]
(j la densité du courant)
avec la 4ème équation de Maxwell on a :
\[
	E\cdot j = \dfrac{1}{\mu_0} rot B \cdot E - \epsilon_0 E \cdot \dfrac{\partial E}{\partial t}
\]
En utilisant :
\[
	div(U \wedge V) = rot (U) \cdot V - U \cdot rot V
\]
Ainsi, on a :
\[
	E \cdot j = -div \left( E \wedge \dfrac{B}{\mu_0} \right) + \dfrac{B}{\mu_0} \cdot rot E - \varepsilon_0 E \cdot \dfrac{\partial E}{\partial t}
\]

À l'aide de la troisième équation de Maxwell 
\[
	E \cdot j = -div \left( E \wedge \dfrac{B}{\mu_0} \right) - \dfrac{B}{\mu_0} \cdot rot E - \varepsilon_0 E \cdot \dfrac{\partial E}{\partial t} 
\]
On définit le vecteur de Poynting R par la relation :
\[
	R = E \wedge \dfrac{B}{\mu_0}
\]

\[
	\vec{E} \cdot \vec{j} = -div \vec{R} - \dfrac{\partial}{\partial t} w
\]
avec \(w = \frac{1}{2} \varepsilon_0 E^2 + \frac{1}{2\mu_0} B^2\)

En abscence de courant on a la loi de conservation des charges :
\[
	div \vec{R} + \dfrac{\partial}{\partial t} w = 0
\]
On peut considérer le vecteur de Poynting cocmme une "densité de courant d'énergie". Il est de la forme :
\[
	\vec{R} = w \vec{v}
\]
où \(\vec{v}\) est la vitesse de propagation de l'énergie

Le vecteur de Poynting a donc pour expression :
\[
	\vec{R} = \vec{E} \wedge \dfrac{(\vec{k} \wedge \vec{E})}{\omega \mu_0}
\]

Finalement :
\[
	\vec{R} = \dfrac{k E^2}{\omega \mu_0} \dfrac{\vec{k}}{k}
\]

La densité d'énergie magnétique peut s'écrire sous la forme :
\[
	W_m = \dfrac{1}{2} \dfrac{B^2}{\mu_0} = \dfrac{1}{2} \varepsilon_0 c^2 B^2
\]

On peut écrire :
\[
	E = \dfrac{\omega}{k} B = cB
\]

La densité d'énergie magnétique est donc égale à la densité d'énergie électrique

\subsection{Impédance cacractéristique}

Dans le cas d'une onde plane monochromatique on définit l'impédance caractéristique d'un milieu par :
\[
	Z = \dfrac{E}{H}
\]
Dans le vide on définit H par :
\[
	\vec{B} = \mu_0 \vec{H}
\]

Donc l'impédance du vide est de 
\[
	Z_0 = 377 \Omega
\]

\end{document}